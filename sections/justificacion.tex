\mychapter{Justificación.}
\addcontentsline{toc}{chapter}{Justificación}

\begin{justify}
    El ciberespacio se considera como un nuevo ámbito de operaciones, pasando de ser una herramienta tecnológica,
    a un campo de batalla donde las \gls{tic}, se convierten en un arma y
    al mismo tiempo en un objetivo de combate \parencite{defensa2022manual}; los drones como hardware móvil dentro de este entorno,
    representan una amenaza híbrida que combina elementos físicos y digitales.\\

    \noindent Diversos casos en México reportados por notas periodísticas como la de \textcite{santos2022drones} del periódico El País,
    evidencian su uso en vigilancia, transporte de contrabando y ataques dirigidos mediante
    el empleo de artefactos explosivos, lo que ha expuesto su capacidad para causar perjuicio a las Fuerzas Armadas.
    De acuerdo con una nota periodística de \textcite{radioformula2024drones}, “… las herramientas que los criminales usan de los drones
    para cometer este delito son que tienen sensores, como cámaras, infrarrojos, \gls{gps}…”.\\

    \noindent Resulta una obligación de desarrollar capacidades humanas, técnicas y tecnológicas en el
    ciberespacio ante amenazas y conflictos de última generación \parencite{defensa2022manual}, en este contexto,
    el desarrollo de un sistema de inhibición de la capacidad de geolocalización de drones mediante señales de
    interferencia GNSS constituye una respuesta eficaz y factible.
\end{justify}