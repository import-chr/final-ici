\phantomsection
\section*{\fontsize{12}{18}\selectfont Software Defined Radio (SDR).}
\addcontentsline{toc}{section}{Software Defined Radio (SDR)}

\begin{justify}
    Conodico también como Software Radio (SR), hace referencia a la comunicación inalámbrica en la cual
    la transmisión modulada es generada o definida por computadora. El objetivo principal de un \gls{sdr}
    es reemplazar tantos componentes analógicos y dispositivos Very Large-Scale Integration (VLSI) digitales
    cableados de transceptores como sea posible por dispositivos programables \parencite{sadiku2004software}.\\

    Es un sistema de radiocomunicaciones en el cual componentes que tradicionalmente se implementaban
    en hardware se realizan mediante software en una computadora o sistema embebido. Esto permite que el mismo
    hardware sea adaptable para cumplir con diversos estándares de comunicación simplemente midificando el software,
    lo que otorga una gran flexibilidad y eficiencia en el desarrollo y actualización de sistemas de radio \parencite{gnuradio}.
\end{justify}