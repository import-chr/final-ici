\phantomsection
\section*{\fontsize{12}{18}\selectfont Navegación por Satélite.}
\addcontentsline{toc}{section}{Navegación por Satélite}

\begin{justify}
    La navegación por satélite es un sistema de posicionamiento global que permite a los usuarios determinar su ubicación y hora en cualquier lugar del mundo.
    El sistema de navegación por satélite más conocido es el \gls{gps}, que fue desarrollado por el Departamento de Defensa de los Estados Unidos.
    El sistema de navegación por satélite utiliza una red de satélites en órbita alrededor de la Tierra para transmitir señales a los receptores de los usuarios.
    Estas señales son utilizadas por los receptores para calcular la posición y la hora del usuario.\\

    El sistema de navegación por satélite es utilizado en una amplia variedad de aplicaciones, incluyendo la navegación de vehículos, la navegación marítima,
    la navegación aérea, la geolocalización de objetos y personas, la cartografía y la topografía, la agricultura de precisión, la gestión de flotas,
    la logística y el transporte, la seguridad y la defensa, la meteorología y la investigación científica.
\end{justify}

\phantomsection
\section*{\fontsize{12}{18}\selectfont Global Navigation Satellite System (GNSS).}
\addcontentsline{toc}{section}{Global Navigation Satellite System (GNSS)}

\begin{justify}
    El término \gls{gnss} se define como el conjunto de todos los sistemas de \gls{satnav} y sus aumentaciones. El GNSS proporciona información precisa, continua y mundial
    sobre la posición tridimensional y la velocidad a los usuarios que cuentan con el equipo receptor adecuado; además, también difunde el tiempo dentro de la escala de \gls{utc}.
    Las constelaciones globales dentro del \gls{gnss}, a veces denominadas constelaciones principales, consisten nominalmente en 24 o más satélites en \gls{meo} dispuestos
    en 3 o 6 planos orbitales con cuatro o más satélites por plano. Una red de monitoreo y control en tierra supervisa la salud y el estado de los satélites.
    Esta red también carga datos de navegación y otra información en los satélites.\\

    Con la excepción del \gls{rdss} proporcionado por una parte del sistema \gls{bds}, que se basa en el cálculo de rangos activos a satélites geoestacionarios para la determinación
    de la posición, estos sistemas \gls{satnav} utilizan el concepto de medición unidireccional del \gls{toa}. Las transmisiones de los satélites están referenciadas a estándares
    de frecuencia atómicos altamente precisos a bordo de los satélites, los cuales están sincronizados con una base de tiempo interna del sistema. Todos los sistemas
    \gls{satnav} transmiten códigos de alcance y datos de navegación en dos o más frecuencias utilizando una técnica llamada \gls{dsss}.\\

    Cada satélite transmite señales con el componente de código de alcance precisamente sincronizado a una escala de tiempo común. Los datos de navegación permiten que el
    receptor determine la ubicación del satélite en el momento de la transmisión de la señal, mientras que el código de alcance permite que el receptor del usuario determine
    el tiempo de tránsito de la señal y, por lo tanto, la distancia entre el satélite y el usuario. Esta técnica requiere que el receptor del usuario también contenga un reloj.
    Para determinar la ubicación tridimensional del receptor utilizando esta técnica, es necesario realizar mediciones TOA a cuatro satélites. Si el reloj del receptor
    estuviera sincronizado con los relojes de los satélites, solo se necesitarían tres mediciones de distancia. Sin embargo, en los receptores de navegación generalmente
    se emplea un reloj de cristal para minimizar el costo, la complejidad y el tamaño del receptor. Por lo tanto, se requieren cuatro mediciones para determinar la latitud,
    la longitud, la altura del usuario y la desviación del reloj del receptor con respecto a la base de tiempo interna del sistema. Si el tiempo del sistema o la altitud se
    conocen con precisión, se requieren menos de cuatro satélites.
\end{justify}