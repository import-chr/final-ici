\phantomsection
\section*{\fontsize{12}{18}\selectfont Navegación por Satélite.}
\addcontentsline{toc}{section}{Navegación por Satélite}

\begin{justify}
    La navegación por satélite es un sistema de posicionamiento global que permite a los usuarios determinar su ubicación y hora en cualquier lugar del mundo.
    El sistema de navegación por satélite más conocido es el \gls{gps}, que fue desarrollado por el Departamento de Defensa de los Estados Unidos.
    El sistema de navegación por satélite utiliza una red de satélites en órbita alrededor de la Tierra para transmitir señales a los receptores de los usuarios.
    Estas señales son utilizadas por los receptores para calcular la posición y la hora del usuario.\\

    El sistema de navegación por satélite es utilizado en una amplia variedad de aplicaciones, incluyendo la navegación de vehículos, la navegación marítima,
    la navegación aérea, la geolocalización de objetos y personas, la cartografía y la topografía, la agricultura de precisión, la gestión de flotas,
    la logística y el transporte, la seguridad y la defensa, la meteorología y la investigación científica.
\end{justify}

\phantomsection
\section*{\fontsize{12}{18}\selectfont Global Navigation Satellite System (GNSS).}
\addcontentsline{toc}{section}{Global Navigation Satellite System (GNSS)}

\begin{justify}
\end{justify}