\phantomsection
\section*{\fontsize{12}{18}\selectfont Navegación por Satélite.}
\addcontentsline{toc}{section}{Navegación por Satélite}

\begin{justify}
    La navegación por satélite es un sistema de posicionamiento global que permite a los usuarios determinar su ubicación y hora en cualquier lugar del mundo.
    El sistema de navegación por satélite más conocido es el \gls{gps}, que fue desarrollado por el Departamento de Defensa de los Estados Unidos.
    El sistema de navegación por satélite utiliza una red de satélites en órbita alrededor de la Tierra para transmitir señales a los receptores de los usuarios.
    Estas señales son utilizadas por los receptores para calcular la posición y la hora del usuario.\\

    El sistema de navegación por satélite es utilizado en una amplia variedad de aplicaciones, incluyendo la navegación de vehículos, la navegación marítima,
    la navegación aérea, la geolocalización de objetos y personas, la cartografía y la topografía, la agricultura de precisión, la gestión de flotas,
    la logística y el transporte, la seguridad y la defensa, la meteorología y la investigación científica.
\end{justify}

\phantomsection
\section*{\fontsize{12}{18}\selectfont Sistema Global de Navegación por Satélite (GNSS).}
\addcontentsline{toc}{section}{Sistema Global de Navegación por Satélite (GNSS)}

\begin{justify}
    El término \gls{gnss} se define como el conjunto de todos los sistemas de \gls{satnav} y sus aumentaciones. El GNSS proporciona información precisa, continua y mundial
    sobre la posición tridimensional y la velocidad a los usuarios que cuentan con el equipo receptor adecuado; además, también difunde el tiempo dentro de la escala de \gls{utc}.
    Las constelaciones globales dentro del \gls{gnss}, a veces denominadas constelaciones principales, consisten nominalmente en 24 o más satélites en \gls{meo} dispuestos
    en 3 o 6 planos orbitales con cuatro o más satélites por plano. Una red de monitoreo y control en tierra supervisa la salud y el estado de los satélites.
    Esta red también carga datos de navegación y otra información en los satélites.\\

    Con la excepción del \gls{rdss} proporcionado por una parte del sistema \gls{bds}, que se basa en el cálculo de rangos activos a satélites geoestacionarios para la determinación
    de la posición, estos sistemas \gls{satnav} utilizan el concepto de medición unidireccional del \gls{toa}. Las transmisiones de los satélites están referenciadas a estándares
    de frecuencia atómicos altamente precisos a bordo de los satélites, los cuales están sincronizados con una base de tiempo interna del sistema. Todos los sistemas
    \gls{satnav} transmiten códigos de alcance y datos de navegación en dos o más frecuencias utilizando una técnica llamada \gls{dsss}.\\

    Cada satélite transmite señales con el componente de código de alcance precisamente sincronizado a una escala de tiempo común. Los datos de navegación permiten que el
    receptor determine la ubicación del satélite en el momento de la transmisión de la señal, mientras que el código de alcance permite que el receptor del usuario determine
    el tiempo de tránsito de la señal y, por lo tanto, la distancia entre el satélite y el usuario. Esta técnica requiere que el receptor del usuario también contenga un reloj.
    Para determinar la ubicación tridimensional del receptor utilizando esta técnica, es necesario realizar mediciones TOA a cuatro satélites. Si el reloj del receptor
    estuviera sincronizado con los relojes de los satélites, solo se necesitarían tres mediciones de distancia. Sin embargo, en los receptores de navegación generalmente
    se emplea un reloj de cristal para minimizar el costo, la complejidad y el tamaño del receptor. Por lo tanto, se requieren cuatro mediciones para determinar la latitud,
    la longitud, la altura del usuario y la desviación del reloj del receptor con respecto a la base de tiempo interna del sistema. Si el tiempo del sistema o la altitud se
    conocen con precisión, se requieren menos de cuatro satélites.
\end{justify}

\phantomsection
\subsection*{\fontsize{12}{18}\selectfont Sistema de Posicionamiento Global (GPS).}
\addcontentsline{toc}{subsection}{Sistema de Posicionamiento Global (GPS)}

\begin{justify}
    Desde su inicio en la década de 1970, el \gls{gps} de Estados Unidos ha evolucionado continuamente. El rendimiento del sistema ha mejorado
    en términos de precisión, disponibilidad e integridad. Esto se atribuye no solo a importantes mejoras tecnológicas en los tres segmentos: espacial, de control y de usuario,
    sino también a la mayor experiencia de la comunidad operativa de la Fuerza Aérea de EE. UU. El \gls{gps} ofrece dos servicios principales: el \gls{pps}
    y el \gls{sps}. El \gls{pps} es un servicio encriptado destinado a usuarios militares y otros usuarios gubernamentales autorizados. El \gls{sps} es gratuito
    y es utilizado por miles de millones de usuarios civiles y comerciales en todo el mundo. Ambos servicios proporcionan señales de navegación para que un receptor de usuario
    determine posición, velocidad y \gls{utc}, referenciado al \gls{usno}.\\
    
    En cuanto al segmento espacial, se han desarrollado hasta la fecha siete bloques de satélites, cada uno con mayores capacidades. Para febrero de 2016, todos los satélites del
    bloque IIF habían sido lanzados. Se planificó el lanzamiento del primer satélite \gls{gps} III para 2018. La constelación nominal del \gls{gps} consta de 24 satélites distribuidos en 6
    planos orbitales de \gls{meo}, conocida como la constelación base de 24 posiciones. Durante muchos años, la \gls{usaf} ha operado la
    constelación con más satélites que el número base, con un diseño expandible para acomodar hasta 27 satélites en posiciones definidas. Esta reconfiguración formalizada de
    hasta 27 satélites ha resultado en una mejor cobertura y propiedades geométricas en la mayoría de las partes del mundo. Los satélites adicionales (más allá de 27)
    generalmente se colocan junto a satélites que se espera necesiten reemplazo en un futuro cercano.
\end{justify}

\phantomsection
\subsection*{\fontsize{12}{18}\selectfont Sistema GLONASS.}
\addcontentsline{toc}{subsection}{Sistema GLONASS}

\begin{justify}
    GLONASS es la contraparte rusa del \gls{gps}. GLONASS proporciona servicios de navegación multifrecuencia en la banda L para soluciones de posición, velocidad y tiempo
    en aplicaciones marítimas, aéreas, terrestres y espaciales, tanto dentro de Rusia como a nivel internacional. La forma de tiempo proporcionada a los usuarios es el \gls{utc}.
    A la fecha (\DTMdisplaydate{2025}{03}{10}{}) GLONASS consta de una constelación de 24 satélites operativos en \gls{meo}, un segmento de control en tierra y equipos de usuario. La constelación GLONASS
    está compuesta por varios tipos de satélites: Glonass-M, una versión modernizada del satélite original lanzada desde 1982 hasta 2005; Glonass-K1, cuyo primer
    lanzamiento fue en 2011; Glonass-K2, a partir de 2018.\\
    
    Tanto los satélites Glonass-M como Glonass-K1 transmiten códigos de alcance cortos y largos, así como datos de navegación utilizando \gls{fdma}. Estos satélites también
    transmiten un código de alcance con datos de navegación mediante \gls{cdma}. Los satélites Glonass-K llevan una carga útil de \gls{sar}, que retransmite las
    transmisiones de balizas \gls{sar} de 406 MHz. GLONASS cuenta con una red de estaciones terrestres
    ubicadas principalmente dentro de las fronteras de Rusia, complementada por estaciones de monitoreo fuera de sus fronteras. El gobierno ruso ha decretado que el servicio abierto
    de GLONASS está disponible para todos los usuarios nacionales e internacionales sin ninguna limitación. Por lo tanto, actualmente está incorporado en receptores GNSS
    multiconstelación de un solo chip utilizados por millones de personas cada día.
\end{justify}

\phantomsection
\subsection*{\fontsize{12}{18}\selectfont Sistema Galileo.}
\addcontentsline{toc}{subsection}{Sistema Galileo}

\begin{justify}
    En 1998, la Unión Europea decidió desarrollar un sistema de navegación por satélite independiente del \gls{gps}, diseñado específicamente
    para uso civil a nivel mundial. El desarrollo del sistema Galileo ha seguido un enfoque incremental, donde cada fase tiene su propio conjunto de objetivos. Las dos principales
    fases de implementación son la fase de \gls{iov} y la fase de \gls{foc}. La fase \gls{iov} se ha completado, proporcionando la validación
    integral de los satélites de Galileo. Tras llevar a cabo una exitosa campaña de validación de servicios a lo largo de 2016, la Comisión
    Europea declaró el inicio de los Servicios Iniciales de Galileo el 15 de diciembre de 2016. Durante el 2017, el sistema se encuentraba en la fase \gls{foc}, que completará el
    despliegue de la constelación de Galileo y la infraestructura terrestre, logrando una validación operativa completa y el rendimiento del sistema. Durante la finalización
    del despliegue, la infraestructura se integrará y probará en construcciones del sistema que contengan versiones de segmentos gradualmente mejoradas, aumentando el número
    de elementos remotos y satélites. Actualmente (\DTMdisplaydate{2025}{03}{10}{}) la constelación de Galileo cuenta con 27 satélites operativos.
\end{justify}