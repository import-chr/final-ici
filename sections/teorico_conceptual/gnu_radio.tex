\phantomsection
\section*{\fontsize{12}{18}\selectfont GNU Radio.}
\addcontentsline{toc}{section}{GNU Radio}

\begin{justify}
    GNU Radio es un kit de herramientas de desarrollo de software libre y de código abierto que proporciona bloques
    de procesamiento de señales para implementar sistemas de \gls{sdr}. Puede utilizarse con hardware de \gls{rf} externo
    de bajo costo para crear \gls{sdr}, o sin hardware en un entorno de simulación. Se utiliza ampliamente en la investigación
    académica, la industria y la comunidad de radioaficionados para apoyar la investigación en comunicaciones inalámbricas.\\

    GNU Radio Companion (GRC) es una interfaz gráfica de usuario que permite a los desarrolladores contruir aplicaciones
    de GNU Radio de manera visual, facilitando el diseño de diagramas de flujo de procesamiento de señales \parencite{gnuradio}.
\end{justify}