\phantomsection
\section*{\fontsize{12}{18}\selectfont Frecuencias de Transmisión.}
\addcontentsline{toc}{section}{Frecuencias de Transmisión}

\begin{justify}
    De acuerdo con \textcite{tomasi2003sistemas} el espectro de frecuencias se subdivide en subsecciones o bandas. Cada una tiene un nombre y sus limietes;
    el espectro total útil de \gls{rf} se divide en bandas de frecuencias más angostas, y algunas de ellas se
    subdividen en diversos tipos de servicios. Las designaciones de banda según el \gls{ccir} se resumen como sigue:\\

    \gls{elf}. Son señales en un intervalo de frecuencias de 3 a 30 Hz, y comprenden las señales de distribución eléctrica
    (60 Hz) y las de telemetría de baja frecuencia.\\

    \gls{vf}. Son señales en un intervalo de frecuencias de 300 a 3000 Hz, incluyen las que se asocian a la voz humana.\\

    \gls{vlf}. Son señales en un intervalo de frecuencias de 3 a 30 kHz, comprenden al extremo superior del intervalo audible humano.\\

    \gls{lf}. Son señales en un intervalo de frecuencias de 30 a 300 kHz, se usan principalmente en la navegación marina y aeronáutica.\\

    \gls{mf}. Son señales en un intervalo de frecuencias de 300 kHz a 3 MHz, su uso principal se encuentra en emisiones de radiodifusión comerciales
    de radio \gls{am} (535 a 1605 kHz).\\

    \gls{hf}. Son señales en un intervalo de frecuencias de 3 a 30 MHz, con frecuencias llamadas ondas cortas.\\

    \gls{vhf}. Son señales en un intervalo de frecuencias de 30 a 300 MHz, uso en radios móviles, emisión comercial en \gls{fm} (88 a 108 MHz) y en la
    emisión de televisión, en los canales 2 a 13 (54 a 216 MHz).\\

    \gls{uhf}. Son señales entre los límites de 300 MHz y 3 GHz, usadas en la emisión comercial de televisión, en los canales 14 a 83 (470 a 890 MHz).\\

    \gls{shf}. Son señales de 3 a 30 GHz, se usan en sistemas de radiocomunicaciones por microondas y satelites.\\

    \gls{ehf}. Son señales de 30 a 300 GHz, casi no se usan en radiocomunicaciones, a excepción de aplicaciones muy complicadas
    costosas y especializadas.\\
\end{justify}