\phantomsection
\section*{\fontsize{12}{18}\selectfont Ruido.}
\addcontentsline{toc}{section}{Ruido}

\begin{justify}
    Se define como ruido eléctrico como cualquier energía eléctrica indeseable que queda entre la banda
    de paso de la señal. Se puede clasificar en dos categorías: correlacionado y no correlacionado. El
    ruido correlacionado sólo existe cuando hay una señal; el ruido no correlacionado está presente siempre,
    haya o no una señal \parencite{tomasi2003sistemas}.
\end{justify}

\phantomsection
\subsection*{\fontsize{12}{18}\selectfont Ruido Blanco.}
\addcontentsline{toc}{subsection}{Ruido Blanco}

\begin{justify}
    El ruido blanco es un tipo de señal aleatoria cuyo valor en diferentes momentos no tiene relación estadística, siendo
    completamente aleatorio. Su característica distintiva es una \gls{psd} constante \parencite{haykin2008communication} en todas
    las frecuencias. Esto quiere decir que todas las frecuencias dentro del ruido tienen la misma potencia y ninguna sobresale
    sobre las demás, resultando en una gráfica espectral plana. Su nombre proviene de una analogía con la luz blanca, que incluye todas
    las frecuencias del espectro visible con intensidad similar. \\

    Dado que la señal es no correlativa en el tiempo, sus valores en distintos momentos son independientes entre sí. Al aplicar
    una descomposición espectral mediante la transformada de Fourier, el ruido blanco revela una distribución uniforme en frecuencia,
    como una línea horizontal continua debido a que teóricamente posee un ancho de banda infinito.
\end{justify}

\phantomsection
\subsection*{\fontsize{12}{18}\selectfont Ruido Gaussiano.}
\addcontentsline{toc}{subsection}{Ruido Gaussiano}

\begin{justify}
    El ruido gaussiano es un tipo específico de interferencia aleatoria que sigue una distribución estadística normal (gaussiana),
    caracterizada principalmente por tener una media igual a cero y una cierta desviación estándar. De acuerdo con \textcite{carlson2010communication} este ruido es conocido en comunicaciones como \gls{awgn}, una forma
    comúnmente utilizada en modelos teóricos para analizar el rendimiento de los sistemas de comunicación.
    El término "aditivo" indica que este ruido se suma directamente a la señal transmitida sin depender de ella; "blanco" implica que su \gls{psd} es constante
    en todo el espectro, lo que quiere decir que afecta por igual a todas las frecuencias. Por su parte, "gaussiano" denota que los valores del ruido se distribuyen
    según la curva de campana característica de una distribución normal.
\end{justify}

\phantomsection
\subsection*{\fontsize{12}{18}\selectfont Interferencia.}
\addcontentsline{toc}{subsection}{Interferencia}

\begin{justify}
    La interferencia es una forma de ruido externo que ocurre cuando las señales de una fuente producen frecuencias fuera de su ancho de banda asignado,
    afectando otras señales de diferentes fuentes. Esto sucede cuando las armónicas o productos cruzados de una señal caen en la banda de paso de un canal vecino,
    generando interferencia \parencite{tomasi2003sistemas}.
\end{justify}