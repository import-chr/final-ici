\phantomsection
\section*{\fontsize{12}{18}\selectfont Python.}
\addcontentsline{toc}{section}{Python}

\begin{justify}
    Es un lenguaje de programación creado por Guido van Rossum a principios de los años 90
    cuyo nombre está inspirado en el grupo de cómicos ingleses "Monty Python". Es un lenguaje
    similar a Perl, pero con una sintaxis muy limpia y que favorece un código legible.\\

    Se trata de un lenguaje interpretado o de scipting, con tipado dinánimco, fuertemente tipado, multiplataforma
    y orientado a objetos \parencite{gonzalez2011python}.

    \begin{itemize}
        \item \textbf{Lenguaje interpretado o de script}\\
        Python es un lenguaje de programación interpretado, lo que significa que su código fuente se ejecuta mediante
        un intérprete sin necesidad de compilarse previamente a lenguaje máquina. Aunque los lenguajes compilados pueden
        ofrecer un mejor rendimiento, los interpretados, como Python, son más flexibles y portátiles. Además, Python se
        traduce a un código intermedio llamado \textit{bytecode} (.pyc o .pyo) en su primera ejecución,
        lo que optimiza su rendimiento en posteriores ejecuciones.
        
        \item \textbf{Tipado dinámico y opcionalmente estático}\\
        Python permite el tipado dinámico, lo que significa que no es necesario declarar el tipo de una variable, ya que este se
        determina en tiempo de ejecución según el valor asignado. Sin embargo, a partir de Python 3.5, se introdujeron las anotaciones
        de tipo (\textit{type hints}), que permiten definir tipos de datos de manera estática, aunque no son obligatorias y el lenguaje
        sigue siendo dinámico en su esencia.
        
        \item \textbf{Fuertemente tipado}\\
        Python es un lenguaje fuertemente tipado, lo que significa que una variable no puede cambiar de tipo implícitamente. Si se necesita
        operar con distintos tipos de datos, es obligatorio realizar conversiones explícitas.
        
        \item \textbf{Multiplataforma}\\
        Python es un lenguaje multiplataforma, compatible con una gran variedad de sistemas operativos como Windows, macOS, Linux y más. Siempre que
        se eviten dependencias específicas del sistema, el mismo código puede ejecutarse en diferentes plataformas sin modificaciones significativas.
        
        \item \textbf{Orientado a objetos y multiparadigma}\\
        Python es un lenguaje de programación orientado a objetos, lo que permite modelar entidades del mundo real mediante clases y objetos. Además,
        admite otros paradigmas de programación, como la programación funcional e imperativa, ofreciendo flexibilidad para adaptarse a diferentes estilos
        de desarrollo.
    \end{itemize}
\end{justify}