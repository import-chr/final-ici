\chapter*{Planteamiento del Problema.}
\addcontentsline{toc}{chapter}{Planteamiento del Problema}

\begin{justify}
    En los últimos años, el uso de drones ha crecido debido a su accesibilidad, versatilidad y en algunos casos costos accesibles.
    Si bien estas tecnologías tienen aplicaciones legitimas como la agricultura, logística y entretenimiento,
    también han sido adoptadas por organizaciones criminales para actividades ilícitas. En México, 
    se han visto casos en los que los drones han sido utilizados para vigilancia, contrabando y,
    en escenarios más críticos, para realizar ataques directos mediante la instalación de artefactos explosivos.\\ 

    \noindent La capacidad de los drones para operar en entornos complejos y su facilidad de adquisición los convierte
    en una amenaza emergente para la seguridad. Estos dispositivos aprovechan sistemas de navegación por satélite,
    como los proporcionados por \gls{gnss}, que les permiten maniobrar con alta precisión.
    Estas mismas características representan un punto de vulnerabilidad que puede ser explotado para neutralizar parte de su
    funcionamiento mediante la generación de señales de interferencia.\\

    \noindent Por lo anterior, se plantea la siguiente pregunta de investigación:
    ¿Cómo se puede interferir el sistema de posicionamiento GNSS de un dron comercial usando \gls{sdr}?
\end{justify}