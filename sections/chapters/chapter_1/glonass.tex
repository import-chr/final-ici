\section{GLONASS}
\label{sec:glonass}

\begin{justify}
    La configuración orbital de GLONASS consiste de una constelación de 24 satélites distribuidos en tres planos orbitales \parencite{glonass_iac}.
    Dichos planos se orientan con una inclinación de $64.8^\circ$ respecto al ecuador; las órbitas se encuentran aproximandamente
    a $19,100$km de altitud teniendo un periodo orbital de $11$ horas, $15$ minutos y $44$ segundos.

    \begin{figure}[H]
        \centering
        \includegraphics[width=0.8\textwidth]{images/chapter_1/glonass/COUNT1_R_LAST_numberSatellites.png}
        \caption{Número de satélites operativos en la constelación GLONASS}
        \label{fig:glonass_numSatellites}
    \end{figure}

    \begin{table}[H]
        \centering
        \renewcommand{\arraystretch}{1.5}
        \setlength{\tabcolsep}{5pt}
        \begin{tabular}{|p{3cm}|c|c|c|c|}
            \hline
            \multicolumn{5}{|c|}{
                \includegraphics[width=14.5cm]{images/chapter_1/glonass/history_glonass_ka.jpg}
            } \\
            \hline
            \rowcolor{lightblue}
            Capacidades & Glonass & Glonass-M & Glonass-K & Glonass-K2 \\
            \hline
            \cellcolor{lightblue}Implementación & 1982-2005 & 2003-2016 & 2011-2018 & 2017+ \\
            \hline
            \cellcolor{lightblue}Estado & Desarmado & En uso & Optimizando & En desarrollo \\
            \hline
            \cellcolor{lightblue}Vida útil (años) & 3.5 & 7 & 10 & 10 \\
            \hline
            \cellcolor{lightblue}Masa (kg) & 1500 & 1415 & 935 & 1600 \\
            \hline
            \cellcolor{lightblue}Señales de acceso abierto & \fontsize{9}{11}\selectfont{L1OF (1602 MHz)} &
            \parbox[t]{2.5cm}{
                \vspace{-7pt}
                {\fontsize{9}{11}\selectfont
                    \begin{itemize}[label={}, leftmargin=0pt, topsep=0pt, partopsep=0pt, parsep=0pt, itemsep=0pt]
                        \item L1OF (1602 MHz)
                        \item L2OF (1246 MHz)
                        \item L3OC (1202 MHz)
                    \end{itemize}
                }
            } &
            \parbox[t]{2.5cm}{
                \vspace{-7pt}
                {\fontsize{9}{11}\selectfont
                    \begin{itemize}[label={}, leftmargin=0pt, topsep=0pt, partopsep=0pt, parsep=0pt, itemsep=0pt]
                        \item L1OF (1602 MHz)
                        \item L2OF (1246 MHz)
                        \item L3OC (1202 MHz)
                        \item L2OC (1248 MHz)
                    \end{itemize}
                }
            } &
            \parbox[t]{2.5cm}{
                \vspace{-7pt}
                {\fontsize{9}{11}\selectfont
                    \begin{itemize}[label={}, leftmargin=0pt, topsep=0pt, partopsep=0pt, parsep=0pt, itemsep=0pt]
                        \item L1OF (1602 MHz)
                        \item L2OF (1246 MHz)
                        \item L1OC (1600 MHz)
                        \item L2OC (1248 MHz)
                        \item L3OC (1202 MHz)
                    \end{itemize}
                    \vspace{10pt}
                }
            } \\
            \hline
            \cellcolor{lightblue}Señales de acceso restringido &
            \parbox[t]{2.5cm}{
                \vspace{-7pt}
                {\fontsize{9}{11}\selectfont
                    \begin{itemize}[label={}, leftmargin=0pt, topsep=0pt, partopsep=0pt, parsep=0pt, itemsep=0pt]
                        \item L1SF (1592 MHz)
                        \item L2SF (1237 MHz)
                    \end{itemize}
                }
            } &
            \parbox[t]{2.5cm}{
                \vspace{-7pt}
                {\fontsize{9}{11}\selectfont
                    \begin{itemize}[label={}, leftmargin=0pt, topsep=0pt, partopsep=0pt, parsep=0pt, itemsep=0pt]
                        \item L1SF (1592 MHz)
                        \item L2SF (1237 MHz)
                    \end{itemize}
                }
            } &
            \parbox[t]{2.5cm}{
                \vspace{-7pt}
                {\fontsize{9}{11}\selectfont
                    \begin{itemize}[label={}, leftmargin=0pt, topsep=0pt, partopsep=0pt, parsep=0pt, itemsep=0pt]
                        \item L1SF (1592 MHz)
                        \item L2SF (1237 MHz)
                        \item L2SC (1248 MHz)
                    \end{itemize}
                }
            } &
            \parbox[t]{2.5cm}{
                \vspace{-7pt}
                {\fontsize{9}{11}\selectfont
                    \begin{itemize}[label={}, leftmargin=0pt, topsep=0pt, partopsep=0pt, parsep=0pt, itemsep=0pt]
                        \item L1SF (1592 MHz)
                        \item L2SF (1237 MHz)
                        \item L1SC (1600 MHz)
                        \item L2SC (1248 MHz)
                    \end{itemize}
                    \vspace{10pt}
                }
            } \\
            \hline
        \end{tabular}
        \caption{Características de las Generaciones de Satélites GLONASS}
        \label{tab:glonass_satellites}
    \end{table}
\end{justify}

\subsection{Banda L1}

\begin{justify}
    \textbf{Características}
    \begin{itemize}
        \item Frecuencia central 1602 $\pm$ 0.5625 MHz.
        \item Ancho de banda \( \sim \) 7.3 MHz (1598.0625 - 1605.375 MHz).
        \item Señal y modulación:
        \begin{itemize}
            \item L1OCd - \gls{bpsk} (1)
            \item L1OCp - \gls{boc} (1, 1)
        \end{itemize}
    \end{itemize}
\end{justify}

\subsection{Banda L2}

\begin{justify}
    \textbf{Características}
    \begin{itemize}
        \item Frecuencia central 1246 $\pm$ 0.4375 MHz.
        \item Ancho de banda \( \sim \) 5.6 MHz (1242.9375 - 1248.625 MHz).
        \item Señal y modulación:
        \begin{itemize}
            \item L2 KSI - \gls{bpsk} (1)
            \item L2OCp - \gls{boc} (1, 1)
        \end{itemize}
    \end{itemize}
\end{justify}

\subsection{Banda L3}

\begin{justify}
    \textbf{Características}
    \begin{itemize}
        \item Frecuencia central 1202.025 MHz.
        \item Señal y modulación:
        \begin{itemize}
            \item L3OCd - \gls{bpsk} (10)
            \item L3OCp - \gls{bpsk} (10)
        \end{itemize}
    \end{itemize}
\end{justify}