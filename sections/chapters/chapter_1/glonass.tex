\section{GLONASS}
\label{sec:glonass}

\begin{justify}
    La configuración orbital de GLONASS consiste de una constelación de 24 satélites distribuidos en tres planos orbitales \parencite{glonass_iac}.
    Dichos planos se orientan con una inclinación de $64.8^\circ$ respecto al ecuador; las órbitas se encuentran aproximandamente
    a $19,100$km de altitud teniendo un periodo orbital de $11$ horas, $15$ minutos y $44$ segundos.

    \begin{figure}[H]
        \centering
        \includegraphics[width=0.8\textwidth]{images/chapter_1/glonass/COUNT1_R_LAST_numberSatellites.png}
        \caption{Número de satélites operativos en la constelación GLONASS}
        \label{fig:glonass_numSatellites}
    \end{figure}

    \begin{table}[H]
        \centering
        \renewcommand{\arraystretch}{1.5}
        \setlength{\tabcolsep}{5pt}
        \begin{tabular}{|c|c|c|c|c|}
            \hline
            \multicolumn{5}{|c|}{
                \includegraphics[width=14.5cm]{images/chapter_1/glonass/history_glonass_ka.jpg}
            } \\
            \hline
            Capacidades & Glonass & Glonass-M & Glonass-K & Glonass-K2 \\
            \hline
            Implementación & 1982-2005 & 2003-2016 & 2011-2018 & 2017+ \\
            \hline
            Estado & Desarmado & En uso & Optimizando & En desarrollo \\
            \hline
            Vida útil (años) & 3.5 & 7 & 10 & 10 \\
            \hline
            Masa (kg) & 1500 & 1415 & 935 & 1600 \\
        \end{tabular}
    \end{table}
\end{justify}