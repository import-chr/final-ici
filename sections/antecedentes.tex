\chapter*{Antecedentes.}
\addcontentsline{toc}{chapter}{Antecedentes}

\begin{justify}
    
    Investigaciones previas han evidenciado que los sistemas de navegación de los drones, basados principalmente en señales GNSS
    (Global Navigation Satellite Systems), presentan vulnerabilidades que pueden ser explotadas para interferir o
    inhibir su funcionamiento. Estudios como los de Humphreys et al. (2012) han demostrado la posibilidad de generar interferencias
    que afectan la precisión y confiabilidad de estas señales, lo que ha llevado a considerar el desarrollo de tecnologías para
    contrarrestar estas amenazas.\\
    
    \noindent Adicionalmente, se ha observado que la implementación de contramedidas tecnológicas es crucial para proteger áreas sensibles y
    operaciones estratégicas. Autores como Tippenhauer et al. (2011) enfatizan que comprender las vulnerabilidades de los sistemas GNSS
    y desarrollar soluciones efectivas es esencial para garantizar la seguridad en entornos donde la integridad de la información y
    la protección de instalaciones críticas son prioritarias.

\end{justify}