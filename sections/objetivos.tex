\chapter*{Objetivos.}
\addcontentsline{toc}{chapter}{Objetivos}

\phantomsection
\section*{\fontsize{12}{18}\selectfont Objetivo General.}
\addcontentsline{toc}{section}{Objetivo General}

\begin{justify}
    Un software que genere señales de interferencia \gls{gnss} puede interrumpir
    el sistema de posicionamiento de drones comerciales,
    limitando su capacidad de navegación y reduciendo así riesgos de uso
    indebido en zonas restringidas.
\end{justify}

\phantomsection
\section*{\fontsize{12}{18}\selectfont Objetivos Específicos.}
\addcontentsline{toc}{section}{Objetivos Específicos}

\begin{justify}
    \begin{enumerate}
        \item Identificar y analizar las características técnicas de las señales
        \gls{gnss} de las constelaciones \gls{gps}, GLONASS, Galileo para sintonizar
        las señales de interferencia generadas mediante software.\\

        \item Diseñar el esquema de generación de señales, especificando 
        cómo se ajustarán los parámetros clave
        (frecuencia, modulación y potencia) mediante bloques de procesamiento
        en GNU Radio integrados con Python.\\

        \item Implementar en el dispositivo HackRF One los esquemas de
        generación propuestos analizando las afectaciones a los sistemas de
        navegación GNSS de las constelaciones seleccionadas.\\

        \item Realizar pruebas en entornos controlados de interferencia a un
        Dron DJI mini 3 usando los esquemas de generación de señales de
        interferencia propuestos, para evaluar las afectaciones a los sistemas
        de navegación y comportamiento del dron cuando estos se encuentran afectados.
    \end{enumerate}
\end{justify}